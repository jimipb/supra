\chapter{Vorbereitung}
Supraleiter zeichnen sich durch eine widerstandslose Leitung unter einer kritischen
Temperatur $T_c$ aus. Schon seit Anfang des 20. Jahrhunderts bekannt ist dies in
Metallen bei sehr tiefen Temperaturen nahe des Nullpunktes. Hochtemperatursupraleiter
konnten dagegen erst 1986 entdeckt werden. Die theoretischen Grundlagen hinter 
diesen Effekten sind zahlreich und zum Teil noch heute ungeklärt. Hier soll ein
kleiner Überblick gegeben werden.

	\section{Theoretische Grundlagen}

    \subsection{Bosonen und Fermionen}
Zum Verständnis der Supraleitung essentiell ist die Unterscheidung der 
Elementarteilchen in Bosonen und Fermionen, sowie die Eigenschaften Dieser.
Für unsere Zwecke genügt es Bosonen und Fermionen zu betrachten.

        \subsubsection{Fermionen}
Als Fermionen werden Teilchen mit halbzahligem Spin ($\frac{1}{2}, \frac{3}{2},
\dots$) bezeichnet. Diese Gruppe der Elementarteilchen, zu der Quarks und 
Leptonen gezählt werden, bildet die Materie.
\vspace{3pt}\\
Für die Supraleitung wichtig ist vor allem das Pauli'sche Ausschlussprinzip, 
welchem Fermionen unterliegen. Es besagt, dass 2 Fermionen am selben Ort nicht
den selben Quantenzustand besetzen dürfen.\\
Dies wird am Beispiel der Elektronen in Atomorbitalen klar. Ohne dieses Prinzip
könnten alle Elektronen im energetisch günstigsten Grundzustand liegen, in Atomen
werden jedoch Besetzungen energetisch höherliegender Zustände beobachtet.
\begin{figure}[h]
    \centering
    \includegraphics[width=0.4\textwidth]{Abb/energieniveaus-elektronen.jpg}
    \caption{Energieniveaus der Elektronen in Atomorbitalen [1]}
    \label{Abb:energieschema}
\end{figure}
In Abbildung \ref{Abb:energieschema} ist die Aufteilung zu sehen. Zuerst werden die
beiden Zustände der ersten Schale besetzt, die sich nur in der Spinausrichtung 
unterscheiden. In den nächsten Schalen werden durch unterschiedliche Drehimpulse 
und Spinausrichtungen der einzelnen Elektronen weitere Zustände besetzt, mit immer
größer werdender Energie.

        \subsubsection{Bosonen}
Als Bosonen werden Teilchen mit ganzzahligem Spin bezeichnet. Elementar treten 
Bosonen nur als Austauschteilchen der elementaren Wechselwirkungen zwischen 
Fermionen auf. Ein Beispiel ist hier das Photon als Überträger der 
elektromagnetischen Kraft. Für unsere Zwecke sind jedoch aus Fermionen
zusammengesetzte Bosonen weitaus wichtiger.
\vspace{3pt}\\
Mehrere Fermionen können durch Wechselwirkungen so aneinander koppeln, dass sie
durch eine gemeinsame Gesamtwellenfunktion beschrieben werden müssen. Einfach
hat dies zur Folge, dass sich Verbunde aus Fermionen, ein bekanntes Beispiel 
hierfür stellen Atomkerne dar, wie Bosonen verhalten. 
\vspace{3pt}\\
Im Gegensatz zu den Fermionen unterliegen Bosonen nicht dem Pauli-Prinzip. Anders
als im obigen Energieschema, ist es also für Bosonen möglich den selben
Energiezustand zu besetzen. Dies wird später für die Theorie der Supraleitung 
essentiell.

	\subsection{BCS-Theorie}
Eine gute theoretische Beschreibung Supraleiter 1. Art bietet die BCS-Theorie. 
Zwar kann durch diese die Hochtemperatursupraleitung auch erklärt werden, das 
Prinzip der Paarbildung bleibt jedoch ungeklärt.

        \subsubsection{Cooper-Paare}
Grundlage der Supraleitung ist die Cooper-Paarbildung der Elektronen im Festkörper.
\vspace{3pt}\\
Es wird angenommen, dass ein Elektron, aufgrund seiner negativen Ladung, eine 
Deformationsspur hinterlässt. 
\begin{figure}[h]
    \centering
    \includegraphics[width=0.7\textwidth]{Abb/deformation.pdf}
    \caption{Deformationsspur hinter einem Elektron}
    \label{Abb:def}
\end{figure}
Siehe hierzu Abbildung \ref{Abb:def}. Ein Elektron zieht die positiv geladenen Kerne
an, was diese zum Schwingen anregt. Nach einer viertel Schwingperiode erreicht die
Konzentration der positiven Ladungen ihr Maximum und weitere Elektronen werden 
angezogen. Durch die große Reichweite dieser Kraft, das erste Elektron ist nach
einer Viertel Schwingperiode schon weit durch den Festkörper gewandert, wird sie
nicht durch die Coulomb-Abstoßung aufgehoben.
\vspace{3pt}\\
Grundlage der BCS-Theorie ist nun die Kopplung zweier Elektronen durch diese 
Wechselwirkung, wodurch der Verbund durch eine bosonische Gesamtwellenfunktion
beschrieben werden muss.

		\subsubsection{Widerstandslose Leitung}
Wie bereits erklärt, können die bosonischen Cooper-Paare nun alle den Grundzustand
einnehmen. Dies führt zu einer den ganzen Festkörper überspannenden 
Bose-Einstein-Wellenfunktion, weshalb die Supraleitung als auch makroskopischer
Quantenzustand beschrieben wird. \\
Die widerstandslose Leitung erklärt sich nun durch die Energielücke der Breite 
$E_g$ zwischen dem Grundzustand der Cooper-Paare und dem Gitter. Eine Wechselwirkung
zwischen Elektronengas und Gitter kann nicht stattfinden, solange die 
Elektronenenergie $E_g$ nicht erreicht. Mit dieser Erklärung leuchtet auch die 
Existenz der kritischen Stromstärke $I_c$ und der kritischen Temperatur $T_c$ ein.
Überschreitet die Elektronenenergie durch thermische Anregung, oder auch durch
einen Stromfluss die Energielücke, so brechen die Cooper-Paare auf und der 
Festkörper geht wieder in den normalleitenden Zustand über.\\

	\subsection{Meisner-Ochsenfeld Effekt}
Eine weitere wichtige Eigenschaft eines Supraleiters ist sein Verhalten im 
Magnetfeld. Dieses wird durch den Meisner-Ochsenfeld Effekt beschrieben.
\begin{figure}[h]
	\centering
	\includegraphics[width=0.6\textwidth]{Abb/meisner.pdf}
	\caption{Magnetfeld um einen Supraleiter vor und nach Abkühlung unter $T_c$}
	%Von Piotr Jaworski, PioM EN DE PL; POLAND/Poznań - Inspiration: Image:EXPULSION.png, Gemeinfrei, https://commons.wikimedia.org/w/index.php?curid=325237
	\label{Abb:meisner}
\end{figure}
Wie in Abbildung \ref{Abb:meisner} zu sehen, wird ein Magnetfeld, das oberhalb der
kritischen Temperatur $T_c$ den Festkörper durchdringt, in der supraleitenden Phase
aus dem Leiter verdrängt wird. Hier wird in einer dünnen Schicht unter der 
Oberfläche ein Abschirmstrom induziert, der die Feldlinien des äußeren Feldes aus
dem Material verdrängt. Der Supraleiter ist also ein idealer Diamagnet. Dieser 
Effekt wird als Meisner-Ochsenfeld Effekt bezeichnet.

		\subsubsection{Typ-I Supraleiter}
Diese Art funktioniert genau wie in Abbildung \ref{Abb:meisner} gezeigt. In der 
supraleitenden Phase wird das Feld aus dem Inneren verdrängt. Die Normalleitung 
wird durch eine Temperatur größer $T_c$, einen Strom größer $I_c$, oder ein
Magnetfeld stärker als die kritische Feldstärke $B_c$ wiederhergestellt. 

		\subsubsection{Typ-II Supraleiter}
Bei diesen Materialien gibt es eine untere kritische Feldstärke $B_{c1}$. Oberhalb
dieser dringt das Magnetfeld in Form von Flusschläuchen, wie in Abbildung 
\ref{Abb:flusschlauch} zu sehen, in den Festkörper ein. Der elektrische Widerstand
bleibt jedoch weiterhin sehr klein.
\begin{figure}[h]
	\centering
	\includegraphics[width=0.6\textwidth]{Abb/flussschlauch.jpg}
	\caption{Flussschläuche im Typ-II Supraleiter oberhalb von $B_{c1}$}
	\label{Abb:flusschlauch}
	% http://slideplayer.org/slide/789280/2/images/19/GL-Theorie:+Konsequenzen.jpg
\end{figure}
Übersteigt die Feldstärke den Wert der kritischen Feldstärke $B_{c2}$, so geht
der Supraleiter endgültig in die normalleitende Phase über.

