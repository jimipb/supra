\chapter{Einleitung}
Die Supraleitung wurde erstmals 1911 vom niederländischen Physiker Heike Kamerlingh
Onnes entdeckt. Eine quantenmechanische Theorie konnte erst 1957 in Form der 
BCS-Theorie, benannt nach den US-amerikanischen Physikern John Bardeen, Leon Neil 
Cooper und John Robert Schrieffer, entwickelt werden.
Die kommerzielle Nutzung wurde dann durch die Entdeckung der 
Hochtemperatursupraleitung 1986 durch den deutschen Physiker Johannes Georg Bednorz
und dem Schweizer Karl Alex Müller möglich. Seit den 2000ern sind diese immer 
häufiger zur elektrischen Stromversorgung in Generatoren und Motoren verwendet.\\
Im Rahmen dieses Versuchs sollen nun Grundzüge der Forschung rund um die Supraleiter
näher kennengelernt und die Theorie dahinter beleuchtet werden. \cite{suprawiki}
