\chapter{Fazit}
Der Versuch bietet gute erste Einblicke in die Tieftemperaturphysik und das Phänomen der Supraleitung. Zudem wird deutlich, welche Herausforderungen die Messung von scheinbar einfachen Versuchsabläufen bei niedrigen Temperaturen mit sich bringt und welche Sicherheitsvorkehrungen im Umgang mit flüssigem Stickstoff und Helium zu beachten sind.

Leider funktionierte bei uns der Hochtemperatur-Teil des Versuchs aufgrund eines Defekts im Messaufbau nicht, was aber zu keinen größeren Problemen in der Auswertung führte, da wir auf Messergebnisse von anderen Gruppen zurückgreifen konnten.

Ein besonderes Highlight des Versuchs ist die Levitation, welche zweifelsfrei zu den beeindruckendsten physikalischen Erscheinungen zählt und dazu animiert, sich noch weiter mit der Thematik der Supraleitung zu beschäftigen.
